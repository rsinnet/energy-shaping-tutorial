\section{Modeling}
\showtoc

\subsection{Modeling Mechanical Systems}

\begin{frame}
  \frametitle{Generalized Coordinates}
  The smallest set of coordinates that can be used to model a system is called the \blue{generalized coordinates}.
\begin{itemize}
  \item A massive particle requires three coordinates, i.e., $$\sQ \in \R^3.$$
  \item A rigid body requires six coordinates, i.e., $$\sQ \in \R^3 \times \SO = \SE.$$
\end{itemize}
Fewer coordinates can be used on systems with constraints.
\end{frame}


\begin{frame}
  \frametitle{Example: Pendulum on a Cart}

  \begin{figure}
    \centering
    \def\svgwidth{0.8\columnwidth}
    \input{figures/cart_spring.eps_tex}
    \caption{A single coordinate, $x \in \R$, can model a cart with mass $M$ connected to a wall by a spring of stiffness $K$.}
  \end{figure}
\end{frame}


\begin{frame}
  \begin{block}{Newton's Second Law}
    By inspection, the controlled system obeys the dynamics
    \begin{align*}
      M {\ddot x} + K x = F.
    \end{align*}
    This method works on simple systems but does not scale well.
  \end{block}
  
  \begin{block}{Lagrangian Method}
  A more elegant formulation is the method of Lagrange which allows one to obtain a dynamic model by looking at the energy of a system.
  \end{block}
\end{frame}

\begin{frame}
  \frametitle{Lagrangian Systems}
  Mechanical systems are defined by:
  \begin{itemize}
  \item Kinetic energy, $T : T\sQ \to \R^+$,\\
  \item Potential energy, $U : \sQ \to \R$,
  \end{itemize}
  which together comprise the Lagrangian,
  \begin{align*}
    \Lagrangian(q, \dot q) = T(q, \dot q) - U(q).
  \end{align*}
  Dynamical motion in such a system with external forcing $F = B(q) u$ is governed by the Euler--Lagrange equation:
  \begin{align*}
    \frac{d}{dt} \pd{\Lagrangian}{\dq} - \pd{\Lagrangian}{\q} = B(q) u.
  \end{align*}
\end{frame}

\begin{frame}
  \frametitle{Using the Lagrangian Method}
  The cart has kinetic energy $$T(x, {\dot x}) = \frac{1}{2} M {\dot x}^2$$ and potential energy $$U(x) = \frac{1}{2} K x^{2}.$$
The Lagrangian is
\begin{align*}
  \Lagrangian(x, {\dot x}) = \frac{1}{2} M {\dot x}^2 - \frac{1}{2} K x^{2}.
\end{align*}
Applying the Euler--Langrange equation gives
\begin{align*}
  M {\ddot x} + K x = F.
\end{align*}
\end{frame}
