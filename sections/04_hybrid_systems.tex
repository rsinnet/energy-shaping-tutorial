\section{Hybrid Systems}
\showtoc

\subsection{Modeling}

\begin{frame}
  \frametitle{Modeling Hybrid Systems}
  \begin{columns}
    \begin{column}{.63\textwidth}
      \begin{definition}
        A \alert{hybrid control system} is a tuple \vspace{-.3cm}
        $$\HCS = \hcsystem, \vspace{-.4cm}$$
        where
        \begin{itemize}
        \item
          $\Domain \subset \mathcal{X}$ is the \blue{domain of admissibility} with state space $\mathcal{X}$,
        \item
          $\ControlSet$ is a set of \blue{admissible controls},
        \item
          $\Guard$ is a \blue{guard} or \blue{switching surface},
        \item
          $\ResetMap$ is a smooth \blue{reset map},
        \item
          $(f, g)$ is a \blue{control system} on $\Domain$: \vspace{-3mm}
          \begin{align*}
            \dot{x} = f(x) + g(x) \, u.
          \end{align*}
        \end{itemize}
      \end{definition}
    \end{column}
    \begin{column}{.4\textwidth}
      \begin{figure}
        \centering
        \includegraphics[width=.9\textwidth]{hsystem}\\
      \end{figure}
      A \alert{simple hybrid system}:\vspace{-.3cm}
      $$\HS = \hsystem$$
    \end{column}
  \end{columns}
\end{frame}

\begin{frame}
  \frametitle{Swing Phase Dynamics}
  \begin{columns}
    \begin{column}{.55\textwidth}
      For Lagrangian $\Lagrangian$ with coordinates
      \begin{align*}
        x = (\q^{T}, \dq^{T})^{T} \in T\ConfigurationSpace,
      \end{align*}
      the dynamic model has the form
      \begin{align*}
        \D(\q) \, \ddq + \C(\q, \dq) \, \dq + \G(\q) = B(\q) \, u,
      \end{align*}
      or $\dot x = f(q, \dot q) + g(q) \, u,$ with
      \begin{align*}
        f(\q, \dq) &= \left(\!\!\begin{array}{c}
        \dq\\
        \D^{-1}(\q) (-\C(\q, \dq) \, \dq - \G(\q))
        \end{array}\!\!\right),\\
        g(\q) &= \left(\!\!\begin{array}{c}
        \mathbf{0}_{m \times m}\\
        \D^{-1}(\q) B(\q)
        \end{array}\!\!\right).
      \end{align*}
    \end{column}\!\!
    \begin{column}{.45\textwidth}
      \begin{figure}
        \centering
        \vspace{-10mm}
        \caption{Physical configuration}
        \includegraphics[width = 1.0\columnwidth]{robot_config}
      \end{figure}
    \end{column}
  \end{columns}
\end{frame}

\begin{frame}
  \frametitle{Impact Dynamics}
  Introduce extended coordinates $\qe = (X, \q^{T})^{T} \in \mathrm{SE}(3) \times \ConfigurationSpace$ which contains the kinematics of a reference point on the robot. Angular momentum balance based on H{\"u}rm{\"u}zl{\"u} and Marghitu:
  \begin{massump}
    \begin{itemize}
    \item Rigid-body plastic impacts
    \item Enough friction to prevent slipping
    \item Motors do not produce impulses
    \item No instantaneous change in configuration, i.e., $\qe^{-} = \qe^{+}$
    \end{itemize}
  \end{massump}
  Under these assumptions, the discrete dynamics satisfies
  \begin{align*}
    \left[\begin{array}{c c}
        \D_{e}(\qe) & -\Jacobian^{T}(\qe)\\
        \Jacobian(\qe) & \mathbf{0}_{3 \times 3}
      \end{array}\right]
    \left[\begin{array}{c}
        \dq^{+}\\
        \delta F(\qe, \dqe)
      \end{array}\right]
    = \left[\begin{array}{c c}
        \D_{e}(\qe) \, \dqe^{-}\\
        \mathbf{0}_{3}
      \end{array}\right].
  \end{align*}
\end{frame}

\subsection{Solutions to Hybrid Systems}
\begin{frame}[t]
  \frametitle{Hybrid Flows}
  A \alert{hybrid flow} of $\HS$ is a tuple
  \begin{align*}
    \chi^{\HS} = \left( \Delta, \mathscr{I}, \mathscr{C} \right),
  \end{align*}
  \vspace{-2em}
  \begin{itemize}
  \item $\Delta = \{0, 1, 2, \ldots\} \subseteq \Nat$ is a finite or infinite \blue{indexing set},
  \item $\mathscr{I} = \{I_{i} \}_{i \in \Delta}$ is a \blue{hybrid interval} where $I_{i} = [\tau_{i}, \tau_{i + 1}]$,
  \item $\mathscr{C} = \{c_{i} \}_{i \in \Delta}$ is a \blue{collection of solutions} of $f$, i.e., ${\dot c}_{i}(t) = f(c_{i}(t))$ $\forall i \in \Delta$.
  \end{itemize}

  \only<1>{
    \vspace{-1em}
    \begin{figure}
      \centering
      \includegraphics[width=0.5\columnwidth]{hybridflow}
    \end{figure}
    \vspace{-1em}
  }

  \only<2>{
    In addition, for every $i, i + 1 \in \Delta$, it is required that
    \begin{enumerate}
    \item $c_{i}(\tau_{i+1}) \in G$ and
    \item $\Delta(c_{i}(\tau_{i+1})) = c_{i+1}(\tau_{i+1})$
    \end{enumerate}
  }
\end{frame}


\begin{frame}
  \frametitle{Periodic Orbits}
  \begin{figure}    
    \centering
    \def\svgwidth{.45\columnwidth}
    \input{figures/hybrid_periodic_orbit.eps_latex}
    \caption{A hybrid periodic orbit. Stability is ascertained using the method of Poincar\'e.}
  \end{figure}
\end{frame}

\begin{frame}
  \frametitle{Conservative Systems}
  For a conservative system, total energy is conserved; i.e.:
  \begin{align*}
    E_{c} &= T(q, \dot q) + U(q)\\
    &= E(q(0), \dot q(0)) = E_{0}
  \end{align*}
  Dynamical motion for such a system relies on the interplay between kinetic and potential energy, which is expressed in the Euler-Lagrange equation,
  \begin{align*}
    \frac{d}{dt} \pd{\Lagrangian}{\dq} - \pd{\Lagrangian}{\q} = 0.
  \end{align*}
\end{frame}


\begin{frame}
  \frametitle{The Simplest Example: Passive Compass-Gait Biped}
  \only<1>{
    \begin{columns}
      \column{1.5in}
      Dynamic Model:
      \begin{align*}
        M(q) \ddot q + H(q, \dot q) = 0
      \end{align*}
      Control Law:
      \begin{align*}
        u = 0.
      \end{align*}
      \column{1.5in}
      \begin{figure}%width=1.0\columnwidth,
        
        \centering
        \def\svgwidth{1.0\columnwidth}
        \input{figures/cg2d-slope-model.eps_latex}
        \vspace{-2em}
        \caption{Compass-gait biped falling down a slope.}
      \end{figure}
    \end{columns}
  }

  \only<2>{
    \begin{figure}
      \includemedia[
        %width=1.0\columnwidth,
        %height=0.5625\columnwidth,
        width=1.0\columnwidth,
        height=0.5\columnwidth,
        addresource=cg2d_slope.mp4,
        activate=pageopen,
        flashvars={source=cg2d_slope.mp4&loop=true&autoPlay=true}
      ]{}{VPlayer9.swf}
      \caption{Stable passive gaits can be found for a range of slopes.}
    \end{figure}    
  }

  \only<3>{
    \begin{figure}
      \centering
      \includegraphics[width=1.0\columnwidth]{energy_cg2d_slope_model}
      \caption{Energy is exchanged between kinetic and potential in form.}
    \end{figure}    
  }
\end{frame}


\begin{frame}
  \frametitle{Nonconservative Systems}
  For a nonconservative system, energy flows out of the system at a rate of $F_{\nc} \cdot dq$. Thus, the following quantity is conserved:
  \begin{align*}
    E_{c} &= T(q, \dot q) + U(q) - \int_{t_{0}}^{t} \! F_{\nc} \cdot \frac{dq}{d\tau} \ d\tau\\
    &= E(q(0), \dot q(0)) = E_{0}
  \end{align*}
  This equation expresses the interplay between kinetic and potential energy and the flow of energy into and out of the system.
\end{frame}


\begin{frame}
  \frametitle{Example: Active Compass-Gait Biped}
  \only<1>{
    \begin{columns}
      \column{1.5in}
      Dynamic Model:
      \begin{align*}
        M(q) \ddot q + H(q, \dot q) = B(q) u.
      \end{align*}
      Controlled Symmetries:
      \begin{align*}
        u &= G(q) - G(\Psi_{\gamma}(q))
      \end{align*}
      where $\Psi : \S \times \R \to \sQ$ rotates the frame of gravity by $\gamma$.

      \column{1.5in}
      \begin{figure}
        \centering
        \def\svgwidth{1.0\columnwidth}
        \input{figures/cg2d-2link-model.eps_latex}
        \vspace{-2em}
        \caption{Compass-gait biped with Controlled Symmetries.}
      \end{figure}
    \end{columns}
  }
  \only<2>{
    \begin{figure}
      \includemedia[
        %width=1.0\columnwidth,
        %height=0.5625\columnwidth,
        width=1.0\columnwidth,
        height=0.5\columnwidth,
        addresource=cg2d_2link_simulation.mp4,
        activate=pageopen,
        flashvars={source=cg2d_2link_simulation.mp4&loop=true&autoPlay=true}
      ]{}{VPlayer9.swf}
      \caption{Passive downhill gaits can be translated to flat ground with Controlled Symmetries.}
    \end{figure}    
  }
  \only<3>{
    \begin{figure}
      \centering
      \includegraphics[width=1.0\columnwidth]{energy_cg2d_slope_model}
      \caption{Energy of the shaped system is conserved.}
    \end{figure}    
  }
\end{frame}

\begin{frame}
  \frametitle{Example: 3-Link Biped}
  \only<1>{
    \begin{columns}
      \column{1.5in}
      Dynamic Model:
      \begin{align*}
        M(q) \ddot q + H(q, \dot q) = B(q) u
      \end{align*}
      Control Law:
      \begin{align*}
        u &= G(q) - G(\Psi(q))\\
        u_3 &=-k_{d} (\dot \vartheta_{T}^{a})\\
        &\hspace{1.8em} -k_{p} (\vartheta_{T}^{a} - \vartheta_{T}^{d})
      \end{align*}
      \column{1.5in}
      \begin{figure}
        \centering
        \def\svgwidth{1.0\columnwidth}
        \input{figures/cg2d-3link-model.eps_latex}
        \vspace{-2em}
        \caption{3-link biped configuration.}
      \end{figure}
    \end{columns}
  }

  \only<2>{
    \begin{figure}
      \centering
      \includegraphics[width=1.0\columnwidth]{energy_cg2d_3link}
      \caption{Energy is not conserved as the controller injects energy.}
    \end{figure}
  }

  \only<3>{
    \begin{figure}
      \includemedia[
        width=1.0\columnwidth,
        height=0.5\columnwidth,
        addresource=cg2d_3link_simulation.mp4,
        activate=pageopen,
        flashvars={source=cg2d_3link_simulation.mp4&loop=true&autoPlay=true}
      ]{}{VPlayer9.swf}
      \caption{Stable passive gaits can be found for a range of slopes.}
    \end{figure}
  }
\end{frame}

\subsection{Types of Systems}
\begin{frame}
  \frametitle{Conservative Systems}
  For hybrid systems with non-dissipative continuous dynamics, the total energy functions as the necessary conserved energy function,
  \begin{align*}
    E_{c}(x) = T(x) + U(x).
  \end{align*}
  This ties in with the method of \blue{Controlled Lagrangians}, increasing the robustness of Lagrangian shaping.
\end{frame}

\begin{frame}
  \frametitle{Example: Compass Gait as a Shaped System}
  \only<1>{
    \begin{columns}
      \column{1.5in}
      Dynamic Model:
      \begin{align*}
        M(q) \ddot q + H(q, \dot q) = B(q) u
      \end{align*}
      Control Law:
      \begin{align*}
        \bar u(q) &= G(q) - G(\Psi(q))
      \end{align*}

      \column{1.5in}
      \begin{figure}
        \centering
        \def\svgwidth{1.0\columnwidth}
        \input{figures/cg2d-2link-model.eps_latex}
        \vspace{-2em}
        \caption{Compass-gait biped with Controlled Symmetries.}
      \end{figure}
    \end{columns}
  }
  \only<2>{
    \begin{figure}
      \centering
      \includegraphics[width=1.0\columnwidth]{energy_cg2d_slope_model}
      \caption{Energy of the shaped system is conserved.}
    \end{figure}    
  }
  %  \only<3>{
  %    \includemedia[
  %      width=1.0\columnwidth,
  %      height=0.5625\columnwidth,
  %      addresource=amber2d.mp4,
  %      activate=pageopen,
  %      flashvars={source=amber2d.mp4&loop=true&autoPlay=true}
  %    ]{}{}%VPlayer9.swf}
  %  }

  \only<3>{
    Energy shaping can be achieved using:
    \begin{align}
      \nonumber
      \argmin_{v = (\delta, u)}  \, & v^T \! H \, v + h^T(q, \dot q) v\\
      \label{clf} \tag{clf}
      \mbox{s.t. } & \Aclf(q, \dot q) v \leq \bclf(q, \dot q)
    \end{align}
    where
    \begin{align*}
      H = \left(\begin{array}{c c}\epsilon & 0\\ 0 & I\end{array}\right), \qquad
        h(q, \dot q) = \left(\begin{array}{c} 0\\ -2 \, \bar u(q, \dot q) \end{array}\right),
    \end{align*}
    and
    \begin{align*}
      \Aclf = \left(\begin{array}{c c}
        -1 & 2 \eta L_{g} \eta
      \end{array}\right), \qquad
      \bclf = -\epsilon \eta^{2} - 2\eta L_{f} \eta
    \end{align*}
  }

  \only<4>{
    \begin{figure}
      \centering
      \includegraphics[width=1.0\columnwidth]{es_comparison_2link_conservative}
      \caption{Demonstration of energy shaping on 2-link biped.}
    \end{figure}
  }

  \only<5>{
    \begin{figure}
      \centering
      \includemedia[
        %width=1.0\columnwidth,
        %height=0.5625\columnwidth,
        width=1.0\columnwidth,
        height=0.5\columnwidth,
        addresource=cg2d_es.mp4,
        activate=pageopen,
        flashvars={source=cg2d_es.mp4&loop=true&autoPlay=true}
      ]{}{VPlayer9.swf}
      \caption{Energy shaping to stabilize to a gait from distant initial condition.}
    \end{figure}
  }
\end{frame}


\begin{frame}
  \frametitle{Nonconservative Systems}
  For hybrid systems with nonconservative continuous dynamics, the energy exchange must be considered, leading to the conserved energy function
  \begin{align*}
    E_{c}(\q, \dq) = T(\q, \dq) + U(q) + \int_{t_{0}}^{t} \! F \cdot \frac{d\q}{d\tau} \ d\tau.
  \end{align*}
  Since the energy dynamics is captured in its entirety, this function can be used for any forced mechanical system.
\end{frame}

\begin{frame}
  \frametitle{Example: 3-Link Biped}
  \only<1>{
    \begin{columns}
      \column{1.5in}
      Dynamic Model:
      \begin{align*}
        M(q) \ddot q + H(q, \dot q) = B(q) u
      \end{align*}
      Control Law:
      \begin{align*}
        {\bar u} &= G(q) - G(\Psi(q))\\
        {\bar u}_3 &=-k_{d} (\dot \vartheta_{T}^{a})\\
        &\hspace{1.8em} -k_{p} (\vartheta_{T}^{a} - \vartheta_{T}^{d})
      \end{align*}
      \column{1.5in}
      \begin{figure}
        \centering
        \def\svgwidth{1.0\columnwidth}
        \input{figures/cg2d-3link-model.eps_latex}
        \vspace{-2em}
        \caption{3-link biped configuration.}
      \end{figure}
    \end{columns}
  }
  \only<2>{
    \begin{figure}
      \centering
      \includegraphics[width=1.0\columnwidth]{energy_conserved_cg2d_3link}
      \caption{The quantity, $E_{0}(\q, \dq) = T(\q, \dq) + V(\q) + \int_{t_{0}}^{t} F_{nc} \cdot \frac{d\q}{d\tau} d\tau$, is conserved.}
    \end{figure}
  }
  \only<3>{
    \begin{figure}
      \centering
      \includegraphics[width=1.0\columnwidth]{es_comparison_3link}
      \caption{Demonstration of energy shaping on 3-link biped.}
    \end{figure}
  }


  \only<4>{
    \begin{figure}
      \centering
      \includemedia[
        %width=1.0\columnwidth,
        %height=0.5625\columnwidth,
        width=1.0\columnwidth,
        height=0.5\columnwidth,
        addresource=3link_es.mp4,
        activate=pageopen,
        flashvars={source=3link_es.mp4&loop=true&autoPlay=true}
      ]{}{VPlayer9.swf}
      \caption{Energy shaping to stabilize to a gait from distant initial condition.}
    \end{figure}
  }

\end{frame}

\begin{frame}
  \frametitle{Multi-Domain Hybrid Systems}
  \begin{itemize}
  \item Complex hybrid systems are made by stitching together domains.
  \item Conserved energy is unique to each domain, i.e.,
    \begin{align*}
      E_0^{1} \to E_{0}^{2} \to \cdots \to E_{0}^{n-1} \to E_{0}^{n} \to E_{0}^{1} \to \cdots
    \end{align*}
  \item Each domain shapes to a specific conserved energy level.
  \end{itemize}
\end{frame}


\begin{frame}
  \frametitle{Example: 7-Link Biped with Feet}
  \only<1>{
    \blue{Appeared in {\bf Automatica}, Jun. 2014}
    \begin{columns}
      \column{1.5in}
      Dynamic Model:
      \begin{align*}
        M(q) \ddot q + H(q, \dot q) = J^{T}(q) \lambda + B(q) u
      \end{align*}
      \column{1.5in}
      \begin{figure}
        \centering
        \def\svgwidth{1.0\columnwidth}
        \input{figures/cg2d-7link-model.eps_latex}
        \vspace{-2em}
        \caption{7-link biped configuration.}
      \end{figure}
    \end{columns}
  }

  \only<2>{
    \begin{block}{Spring--Damper Controller}
      \begin{align*}
        u(\q) = -\beta e^{-\rho \, h_{\mathrm{nst}}(\q)}
      \end{align*}
      where $\beta, \rho \in \R^{+}$ and $h_{\mathrm{nst}} : \ConfigurationSpace \to \R$ is the height of the Nonstance toe.
      Used on
      \begin{itemize}
      \item Stance ankle\\
      \item Nonstance ankle\\
      \item Absolute torso angle\\
      \item Non-stance knee spring (only in double support)
      \end{itemize}
      \blue{Provides passivity-based control and toe-off.}
    \end{block}
  }

  \only<3>{
    \begin{block}{Scuffing Prevention Controller}
      \begin{align*}
        u(\q) = k_p \theta(\q) - k_d \dot \theta(\dq)
      \end{align*}
      where $k_{p}, k_{d} \in \R^{+}$. Used on
      \begin{itemize}
      \item Nonstance ankle
      \end{itemize}
      \blue{Prevents the nonstance toe from colliding with the ground prematurely.}
    \end{block}
  }

\only<4>{
  \begin{figure}
    \centering
      \includegraphics[height=.7\textheight]{domaingraph}
      \caption{The resulting gait traverses a four-domain directed graph.}
    \end{figure}
  }

  \only<5>{
    \begin{figure}
      \centering
      \includemedia[
        %width=1.0\columnwidth,
        %height=0.5625\columnwidth,
        width=1.0\columnwidth,
        height=0.5\columnwidth,
        addresource=7link_es.mp4,
        activate=pageopen,
        flashvars={source=7link_es.mp4&loop=true&autoPlay=true}
      ]{}{VPlayer9.swf}
      \caption{Energy shaping to stabilize to a gait.}
    \end{figure}
  }

  \only<6>{
    \begin{figure}
      \centering
      \includegraphics[width=1.0\columnwidth]{energy_conserved_7link}
      \caption{The quantity, $E_{0} \equiv T(q, \dot q) + V(q) + \int_{t_0}^{t} F_{nc} \cdot \frac{d\q}{d\tau} d\tau$, is conserved.}
    \end{figure}
  }

  \only<7>{
    \begin{figure}
      \centering
      \includegraphics[width=1.0\columnwidth]{energy_conserved_7link_outline}
      \caption{The quantity, $E_{0} \equiv T(q, \dot q) + V(q) + \int_{t_0}^{t} F_{nc} \cdot \frac{d\q}{d\tau} d\tau$, is conserved.}
    \end{figure}
  }
  \only<8>{
    \begin{figure}
      \centering
      \includemedia[
        %width=1.0\columnwidth,
        %height=0.5625\columnwidth,
        width=1.0\columnwidth,
        height=0.5\columnwidth,
        addresource=7link_noes.mp4,
        activate=pageopen,
        flashvars={source=7link_noes.mp4&loop=true&autoPlay=true}
      ]{}{VPlayer9.swf}
      \caption{Many steps on the limit cycle.}
    \end{figure}
  }
\end{frame}
